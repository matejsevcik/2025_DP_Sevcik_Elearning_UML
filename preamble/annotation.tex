\newpage
\thispagestyle{empty}
\vspace*{\fill}
%Čestne vyhlasujem, že som túto prácu vypracoval samostatne, na základe konzultácií a s použitím uvedenej literatúry.
I declare in my honor that I have prepared this work independently, on the basis of consultations and using the mentioned literature.

\vspace{2cm}
\mbox{}
In Bratislava, \myDate \hfill \myName
\mbox{}




\newpage
\thispagestyle{empty}
\mbox{}
\newpage

\thispagestyle{empty}




\thispagestyle{empty}
\section*{Annotation}

\begin{minipage}[t]{1\columnwidth}%
Slovak University of Technology Bratislava 

Faculty of Informatics and Information Technologies

Degree Course: \myStudyProgram\\

Author: \myName

Diploma Thesis: \myTitle

Supervisor: \mySupervisor

\myDate%
\end{minipage}

\bigskip{}

%\lipsum[3]
E-learning is a perspective field that offers many opportunities to support education. It has the potential to increase the effectiveness of learning, reduce the burden on educators, make education more accessible to the general public, and introduce innovative practices that traditional educational approaches do not allow. Despite this, e-learning is used relatively rarely in Slovak schools, even at faculties of informatics. In this project, we focused on the possibilities of applying e-learning to teaching class diagrams, with the potential for expansion to other UML diagrams. We analyzed various techniques used in e-learning and their application in teaching software modeling. The main goal of this project is to create a tool for teaching class diagrams. We applied practices that, according to our analysis, achieved the best results, and supplemented them with our own ideas and adapted them to the needs of the platform. We focused on testing the knowledge of students who used our tool. The main contribution of the project is that our tool, along with the analysis and tests, has created a foundation for further research and improvement of the application of e-learning in the field of software modeling.

Keywords: e-learning, class diagram, software modeling

\newpage{}\thispagestyle{empty}

\newpage
\thispagestyle{empty}
\mbox{}
\newpage

\thispagestyle{empty}
\section*{Anotácia}

\begin{minipage}[t]{1\columnwidth}%
Slovenská technická univerzita v Bratislave

Fakulta informatiky a informačných technológií

Študijný program: \myStudyProgram\\

Autor: \myName

Diplomová práca: E-Learning modelovania SW pomocou diagramu tried

Vedúci diplomového projektu: \mySupervisor

\myDate%
\end{minipage}

\bigskip{}

%\lipsum[3]

E-learning je rýchlo sa rozvíjajúca oblasť poskytujúca veľa možností podpory vzdelávania. Má potenciál zvýšiť efektivitu vzdelávania, znížiť záťaž na pedagógov, sprístupniť vzdelávanie širšej verejnosti alebo zaviesť inovatívne praktiky, ktoré tradičné vzdelávacie prístupy neumožňujú. Napriek tomu je na slovenských školách e-learning využívaný pomerne zriedkavo, a to aj na fakulte informatiky. V projekte sme sa zamerali na možnosti aplikovania e-learningu pri vyučovaní diagramu tried s možnosťou rozšírenia na ďalšie UML diagramy. Vykonali sme analýzu rôznych prístupov používaných v e-learningu a ich využitia pri vyučovaní modelovania softvéru. Hlavným cieľom tohto projektu je vytvorenie nástroja na výučbu diagramu tried. Aplikovali sme pri tom praktiky, ktoré podľa analýzy v predchádzajúcich experimentoch dosiahli najlepšie výsledky, pričom sme ich doplnili
o vlastné nápady a prispôsobili potrebám náášho nástroja. Dôraz sme kládli na testovanie vedomostí študentov, ktorí náš nástroj využívali. Prínos projektu
spočíva najmä v tom, že náš nástroj, analýza a testy pripravili priestor na ďalšie rozširovanie a vylepšovanie možností e-learningu v oblasti modelovania
softvéru.

Kľúčové slová: e-learning, diagram tried, modelovanie softvéru

\newpage{}\thispagestyle{empty}\medskip{}


\newpage{}

\newpage
\thispagestyle{empty}
\mbox{}
\newpage

\newpage
\thispagestyle{empty}

\thispagestyle{empty}
% \includepdf[pages=1,scale=0.8]{assets/zadanie_en.pdf}
\includepdf[pages=-]{assets/zadanie_sk.pdf}
% \todo{toto by malo byt v anglictine, nedari sa mi najst odkial to treba stiahnut. Tiez mala by tam byt ta legenda pod ciarov?}
\newpage

\thispagestyle{empty}
\mbox{}
\newpage

