\chapter{Introduction}

\pagenumbering{arabic}

% \cite{dynabook} 
% \cite{distractions1,distractions2,interruptions}
% \cite{forgetting}

% Nejaka uvodna omacka

% -uml diagrams are an important part of curriculum at it unis
% -class diagrams is the most important uml diagram
% -how uml is taught in FIIT STU, and why it would be better to have an elearning tool (or tools)
% ...

\ac{uml} is a widely used modeling language. It is used for describing systems in an abstract way and from various perspectives, which makes the ability to understand it and use it a valuable asset for any professional in the field of information technologies. 

Most developers claim to possess knowledge of \ac{uml}. However, in many cases, their knowledge of the topic is rather limited or even inaccurate\footnote{Donald Bell, "An Introduction to the Unified Modeling Language," \textit{IBM Developer}, https://developer.ibm.com/articles/an-introduction-to-uml/, accessed February 23, 2025.}. This observation highlights the need for improving the approach to teaching of software modeling and \ac{uml} in educational institutions. 

Traditional classes often do not provide enough time and space for an actual teacher to give the necessary amount of attention to each of the students. The answer to this problem might lie in electronic learning, or e-learning. It is a practice in which learning material is delivered through digital technologies.

E-learning can offer students interactive materials with automated immediate personalized feedback, structured and easily accessible learning materials or study and exercise recommendations, while simultaneously allowing teachers to spend more time consulting students instead of explaining the basic concepts to them.

E-learning also provides possibilities for increasing students' engagement in the study materials by introducing gamification elements or a sense of competition to the courses.

The class diagram is the most prominent diagram in \ac{uml}. It is also common for students to have trouble using it properly. For these reasons, we decided to aim this study at designing an e-learning course that would act as complementary material to a software modeling university subject. 

Our course will implement a number of modern e-learning approaches to improve information retention, engagement, and the overall learning experience of the students. It will be tested on a sample of students at \ac{fiit}. The results will then be analyzed to determine the effectiveness and further potential of such tools in the process of education of software modeling. 

% Ján Novák, "Výhody online vzdelávania", Eduweb.sk, https://www.eduweb.sk/vyhody-online-vzdelavania, citované 24. mája 2025.
% ...

% Teaching \ac{uml} diagrams at schools usually requires the students to create and submit a number of diagrams. The submissions then need to be evaluated by the teachers, which is very time-consuming. It is also possible that the classes do not provide the time and space for the teachers to give detailed feedback to the students, losing much potential for learning. This indicates a need for tools that would aid in teaching the diagrams, such as online courses or tools for automated assessment of the students' diagrams. 

% ...

% Since most studies in this field are aimed at creating tools for automated assessment of submitted class diagrams \todo{citations}, we decided to design an e-learning module for teaching the \ac{uml} class diagram. After all, drawing diagrams is not the first step a learner should take. Our module should be extendible to include other types of \ac{uml} diagrams in the future.

% ...

