Since the experience of studying computer science contains a large amount of self-study in any case, there are many online resources and courses in the field. Some of them utilize the techniques mentioned in the previous section. We investigated how they were used in more detail to determine whether such practices could be relevant for our intended course. 

\subsection{Gamification}

Gamification is the most commonly mentioned practice when it comes to research of software modeling e-learning, and there are numerous attempts at applying it in the context of teaching the \ac{uml} class diagram. The goal is to increase the students' interest and motivation in the given tasks by exploiting their natural drives. 

In \textit{Actionable Gamification: Beyond Points, Badges, and Leaderboards}~\cite{chou2019actionable}, Yu-kai Chou introduces the Octalysis Framework, a comprehensive model for analyzing and designing engaging experiences through gamification. The framework is built around eight core drives that motivate human behavior, such as meaning, accomplishment, empowerment, ownership, social influence, scarcity, unpredictability, and avoidance. Unlike traditional gamification approaches that focus mainly on superficial rewards, Octalysis emphasizes the psychological factors behind user motivation, enabling the creation of more meaningful and effective game-like experiences in both digital and non-digital contexts.\footnotemark[1]
%\footnotetext{This text was made with the use of \ac{genai}}

The practice of gamification involves several techniques. Cagnazzo et al.\cite{cagnazzo2023umlegend} mark a student's accomplishment by levels - the student starts at level one and is awarded experience points for completing exercises. The amount of experience points gained is affected by the difficulty of the given exercise as well as the number of mistakes made by the student, providing additional motivation to reduce the mistakes for the student. After gaining enough experience points, the student's level increases. The study proposes even the introduction of quests to further engage the learners.

The same study also implements a customizable avatar that represents the student in the application. They can alter the avatar's appearance by changing its clothes, hairstyle and other visual features. Completing exercises unlocks new styling options, again increasing the student's motivation. Lastly, the tool immediately displays feedback to the student by highlighting mistakes in various colors and by changing the student's avatar's facial expression based on the student's performance.

In another work, Jurgelaitis et al.\cite{jurgelaitis2019implementing} implement, in addition to previously mentioned levels and experience points, badges that are awarded for accomplishing specific goals and coins that can be exchanged for certain items. The application displays immediate feedback after completing tasks and visualizes the student's progress. Leaderboards with students ranked by their experience points are also included, invoking the students' natural competitiveness.

There are other options to incorporate the concepts of gamification into the teaching of modeling. Besides some of the already described techniques, Bucchiarone et al.\cite{BUCCHIARONE2023102974} connect modeling with the hangman game - each mistake draws another part of the hangman, until either the exercise is completed or the hangman picture is finished (which means the player has lost).

% \todo{Keby trebalo filler este tuto su opisy gamification ref je z umlegend: Jurgelaitis et al. [8] describe the gami cation of a UML university course with a level-based structure where new topics can only be obtained by increasing one’s level.
% Lastly, LearnER [9] implements a gami ed editor for UML class diagrams as well as Entity-Relationship diagrams using common elements such as points awarded for correctly solved exercises,…
% }


\subsection{Microlearning}

% Though we did not find any studies that would explicitly mention microlearning applied to teaching the class diagram, there are studies that mention it indirectly or studies that do not focus on \ac{uml}, 

Microlearning is the practice of breaking the study material down into small, bite-sized pieces so as to avoid overwhelming the learners. Ramle et al.\cite{ramle2021uml} mention that the Learn section of their application divides the study materials into smaller sections. The general reception of the application was positive and students claimed that it helped them understand the concepts better. That suggests that applying microlearning to the topic of \ac{uml} may improve the students' learning experience and possibly their resulting knowledge as well.

However, it should be noted that, even though microlearning improves motivation and knowledge retention, it is more suitable for early stages of learning when basic concepts are explained. It might not be the ideal approach when tackling more complex concepts or when constructing the bigger picture out of the smaller lessons\footnote{Christopher Pappas. "Microlearning in Online Training: 5 Advantages and 3 Disadvantages." \textit{eLearning Industry}, March 15, 2016. Available at: \url{https://elearningindustry.com/microlearning-in-online-training-5-advantages-and-3-disadvantages}. Accessed May 15, 2025.}. Therefore, microlearning should not be applied to every concept, but rather to the earlier lessons only.

\subsection{Social learning}

Ramollari et al.\cite{trapp2011collaborative} designed a system in which students were encouraged to discuss and share their solutions to modeling exercises. Besides discussions and comments, the students could attach their solutions to other students' work as hints, alternatives or suggestions for improvement. The efforts were evaluated positively.

% In addition to that, even though not an e-learning tool, Shmallo et al.\cite{shmallo2020constructive} also observed that evaluating each other's work improved the modeling skills of students.

% Improving Delivery of UML Class Diagrams Concepts in Computer Science Education Through Collaborative Learning

% Constructive Use of Errors in Teaching the UML Class
% Diagram in an IS Engineering Course