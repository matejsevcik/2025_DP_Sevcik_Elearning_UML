% \todo{a text about the general things like isolation and  and digital literacy and availability of technologies}

Traditional learning refers to the model where a teacher or a tutor presents information and exercises to a group of people in a classroom. This model is typical of educational institutions such as universities. This approach has been utilized for a very long time and it is the standard way of teaching. 

E-learning offers a number of improvements over the traditional model. Unlike traditional learning, online courses can be taken from anywhere and at any time, removing much of the geographic and time constraints for the learners. It is also significantly cheaper, since it removes the need for a teacher or a learning space. With interactive exercises and materials, e-learning can increase student engagement and interest in the topic.

On the other hand, it also presents a set of challenges that make traditional learning still the preferable choice in most cases. The most obvious drawback is that the self-directed nature of e-learning requires a level of discipline from the learners which many lack. Other problems include a lack of social interactions with other students, a lack of personalized feedback from a live instructor and the fact that some people do not have access to the technology necessary to utilize e-learning materials. For some people, digital literacy may also be a problem stopping them from using e-learning resources.

% Hybrid, or blended, learning strives to bring the best of both worlds together by applying digital technologies in live classes. It may  This approach will be dicussed in later sections in more detail.

In a study performed by Ramle et al.\cite{ramle2021uml} a \ac{uml} course was introduced. Students that took the course were tasked by filling out a questionnaire about their experience. The reception was mostly positive, with students making claims such as \emph{``makes me understand more''} or \emph{``makes you get better at UML.''}

Jurgelaitis et al.\cite{jurgelaitis2019implementing} found that the introduction of gamified mechanics to a university \ac{uml} course increased both students' motivation and their average scores. This confirms the general expectations from other works that have not yet collected the necessary data for evaluation. 

Lastly, Soler et al.\cite{soler2010formative} proposed a tool for automated correction of class diagram exercises. The researchers conducted an experiment where students were divided into 2 equally sized groups: one solved a set of exercises using the tool, and the other solved them manually. The number of students who solved all of the exercises correctly was higher in the group that used the tool, and this group also had better average score in a subsequent exam. 

These studies clearly demonstrate that tools can be developed to increase not only the students' knowledge, but also their engagement and overall satisfaction with their educational process. 