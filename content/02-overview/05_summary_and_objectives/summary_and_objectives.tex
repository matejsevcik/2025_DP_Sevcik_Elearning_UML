\ac{uml} is a language used for modeling systems from various perspectives. The class diagram is the most prominent of the \ac{uml} diagrams. It is used in various stages of system development as well as for documentation purposes. 

For these reasons \ac{uml} is a common part of the curriculum at universities related to computer science. Despite its inclusion, it is often observed that students as well as graduates and even professionals struggle with \ac{uml}. To solve this problem, it is a worthwhile effort to improve the education of modeling in universities.

Attempts have been made to improve the situation with the use of e-learning - the use of digital technologies to deliver educational content. E-learning can provide numerous benefits such as increased student engagement, instant personalized feedback or saving the time and attention of teachers. It is also possible to use e-learning as an addition to traditional classrooms rather than their replacement - such practice is called blended learning. 

There are numerous commonly used e-learning practices with various intended benefits. The ones most suitable for educational content related to software modeling are gamification, microlearning, blended learning and social learning.

We analyzed several studies that implemented these techniques in \ac{uml} learning tools. The most popular technique was gamification, where the tools support functions resembling games. Examples are giving experience points or some sort of currency for completing exercises and allowing students to create customizable avatars to represent them in the learning environment.

Most of the studies were aimed at creating tools to enhance the in-person classes at universities, not replace them. Some of them presented courses with study materials, practical examples and tests; some introduced new collaborative learning activities, and some studies proposed tools for automated diagram assessment.

The results of these studies were predominantly positive, showing improvements both in students' scores and their engagement with the course topics. Another benefit of the described tools is that they free up the educators' time and energy that they would usually spend reexplaining concepts and correcting exercises.

These findings clearly document the potential of introducing e-learning tools to enhance the teaching of \ac{uml} in university courses related to software modeling. 

Based on the analysis, we decided to create an online course aimed at the \ac{uml} class diagram. It will implement some of the e-learning techniques, most notably gamification and microlearning, and will test it out on students of a software modeling subject at \ac{fiit}. Our hypothesis is that taking such a course will improve the students' knowledge and understanding of the \ac{uml} class diagram. Should we prove the hypothesis correct, further research can be performed to extend the scope of the course to include other types of diagrams and software modeling materials.

