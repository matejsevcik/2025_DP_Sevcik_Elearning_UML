\section{Common issues in Learning UML}

There are several common challenges students face when learning \ac{uml}. Some of them are not encountered by students exclusively but also by graduates and practitioners. 

% people claim to know uml, but usually know only very little or wrong, citation in the introduction

Fresh computer science graduates are expected to possess at least a basic understanding of the use of \ac{uml} in software development, but that is not often the case \cite{kaur2023improving}. In addition to that, even more experienced practitioners or programmers commonly claim to know \ac{uml}, but later prove to only have a very limited or even wrong understanding of the concepts\footnote{Donald Bell, "An Introduction to the Unified Modeling Language," \textit{IBM Developer}, https://developer.ibm.com/articles/an-introduction-to-uml/, accessed February 23, 2025.}. 

This gap arises because UML is widely recognized as complex and difficult for novices to grasp. Studies have consistently shown that students struggle with selecting the appropriate level of abstraction, applying correct syntax and semantics, and translating mental models into UML diagrams. Moreover, the same recurring mistakes continue to be observed over the years, indicating persistent conceptual misunderstandings and difficulties in mastering the modeling process itself \cite{reuter2020insights}\footnotemark[1]
% \footnotetext{This text was made with the use of \ac{genai}}


There are multiple research papers describing the process of identifying errors made in class diagrams. The studies \cite{shmallo2020constructive}, \cite{chren2019mistakes}, \cite{reuter2020insights} made catalogues of such mistakes, allowing easier identification of errors and potentially using them as educational materials. Chren et al.\cite{chren2019mistakes} revealed that the most common mistakes were related to associations, lack of domain understanding, applying inappropriate levels of detail, placing operations into wrong classes or incorrect use of inheritance. 

Reuter et al.\cite{reuter2020insights} provided a comprehensive taxonomy of problems encountered by students, including confusion between class diagrams and other UML diagram types, difficulties in determining appropriate attributes, operations, or classes, uncertainty in modeling associations and their properties, and problems distinguishing between relationships such as inheritance, realization, aggregation, and composition\footnotemark[1].
% \footnotetext{This text was made with the use of \ac{genai}}

These studies highlight the need to improve the way \ac{uml} is taught in most universities, as the students' knowledge often appears insufficient. Areas for improvement include providing more practical exercises, immediate feedback, increasing students' engagement, or connecting modeling with coding.

% \todo{Toto je zaujimava studia - Revealing Students' UML Class Diagram Modelling Strategies with WebUML and LogViz}