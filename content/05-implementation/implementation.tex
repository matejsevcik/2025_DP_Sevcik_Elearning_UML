\chapter{Implementation}


\section{Nonfunctional requirements - Moodle}

-platform moodle (it does everything, compare its capabilities with requirements)\\
-briefly mention all the other things that go into functional reqs, but do not describe them in more detail

\begin{itemize}
    \item Easy to learn and use
    \item Open source or free to use
    \item Capable of storing and retrieving student data
    \item Scalable and maintainable for future expansion
    \item Easily extensible (adding new features/content)
    \item Restricted access - allw access only to students and teachers at \ac{fiit}
    %(moodle cloud or private servers? Both should work though, at the very worst we can generate accounts for all volunteer students of ours)
\end{itemize}


\section{Functional requirements}

\subsection{E-learning techniques}
\subsection{Theoretical content}
\subsection{Exercises}
\subsection{Feedback questionnaire}
\subsection{Testing and storing students' results}

\section{Research questions}

\textbf{What are the most popular and most successful e-learning practices?}

\textbf{Which of these e-learning practices are applicable to a software modeling course?}

\textbf{How can students' skills in software modeling be effectively evaluated?}


% \item What are the shortcomings of current software modeling education practices, and can they be reduced through the use of e-learning?

\textbf{Is it worthwhile to increase the use of e-learning at \ac{fiit} in courses related to software modeling?}



\section{Deviations from design}
\section{How they can be implemented}

\section{Deployment}

\todo{Deployment - only after winter submission}

