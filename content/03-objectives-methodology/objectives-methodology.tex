\chapter{Objectives and methodology}

\section{Objectives}

The goal of this study is to create an online course for teaching the \ac{uml} class diagram. The target audience is computer science students taking a software modeling course at \ac{fiit}. %The success of the proposed online course will be decided by tests and questionnaires.

It is an introductory project to determine whether it is viable to teach modeling with the use of e-learning. We want this study to be a starting point for further educational content at \ac{fiit} should this project prove successful. In such a case, other researches could extend the course by additional features and content and, most importantly, other types of diagrams.

We present the following research questions based on our initial analysis and objectives:

\begin{enumerate}
\item What are the most popular and most successful e-learning practices?
\item Which of these e-learning practices are applicable to a software modeling course?
\item How can students' skills in software modeling be effectively evaluated?
% \item What are the shortcomings of current software modeling education practices, and can they be reduced through the use of e-learning?
\item Is it worthwhile to increase the use of e-learning at \ac{fiit} in courses related to software modeling?
\end{enumerate}

% Povodne vyzkumne otazky;

% Aké sú najúspešnejšie a najpopulárnejšie praktiky pri e-learningu?
% Ktoré z nich je vhodné aplikovať na vyučovanie modelovania softvéru?
% Ktoré z nich chceme použiť pri vytváraní nášho prototypu?
% Akým spôsobom ohodnotiť znalosti študentov v oblasti modelovania softvéru?
% Aké sú nedostatky v tradičnom prístupe k vyučovaniu? Dali by sa zmenšiť použitím e-learningu?
% Oplatí sa zvyšovať využitie e-learningu na školách s tradičným prístupom
% k vyučovaniu?


\section{Methodology}

The course will be implemented in an established e-learning platform. It will contain theoretical materials, interactive exercises, self-assessment components, as well as e-learning practices based on our analysis. 

The course will be taken by second-year students of computer science who will be presented with a test and a questionnaire to assess the impacts of the course. The test will be made up of theoretical questions and practical exercises, while the questionnaire will be aimed at the overall reception by the students. The most important questions will target the students' interest in such materials and whether they think it is an improvement over the standard way of teaching software modeling.

% This will need revision some day
The students will be divided into two groups: one that used our materials and one that was instructed to search for materials on their own. Both groups will be given the same tests to determine their knowledge after the preparation time.

% We will measure the overall reception by the students, the impact on their knowledge, and search for potential improvements. 